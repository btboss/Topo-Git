\usepackage[utf8]{inputenc}       % LaTeX, comprends les accents !
\usepackage[T1]{fontenc}          % Police contenant les caractères français
\usepackage[francais]{babel}      % Placez ici une liste de langues, la
							      % dernière étant la langue principale
\usepackage{fullpage}			  % Permet de dimunuer les marges de base de latex
							      
\usepackage{float}
\usepackage[xindy]{glossaries}
\makeglossaries

\usepackage{graphicx}		      % pour inclure des graphiques
\usepackage{fullpage}
\usepackage{eso-pic}
%\usepackage[left=3.5cm,right=2cm,top=2cm,bottom=2.5cm] ou [a4paper]{geometry}   % Réduire les marges
%\pagestyle{headings}             % Pour mettre des entêtes avec les titres
                                  % des sections en haut de page
                                  
\usepackage{listings}             %pour insérer du code source
 
\newcommand{\HRule}{\rule{\linewidth}{0.5mm}}
 
\newcommand{\blap}[1]{\vbox to 0pt{#1\vss}}
\newcommand\AtUpperLeftCorner[3]{%
  \put(\LenToUnit{#1},\LenToUnit{\dimexpr\paperheight-#2}){\blap{#3}}%
}
\newcommand\AtUpperRightCorner[3]{%
  \put(\LenToUnit{\dimexpr\paperwidth-#1},\LenToUnit{\dimexpr\paperheight-#2}){\blap{\llap{#3}}}%
}
\title{\LARGE{Topo Git}}           		% Les paramètres du titre : titre, auteur, date
\author{JL HABERBUSCH}			% on peut ajouter \and Co-auteur
\date{\today}                   % La date n'est pas requise (la date du
								% jour de compilation est utilisée si rien d'autres n'est spécifié)
\makeatletter